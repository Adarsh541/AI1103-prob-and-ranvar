
  
\documentclass[journal,12pt,twocolumn]{IEEEtran}

\usepackage{setspace}
\usepackage{gensymb}
\singlespacing
\usepackage[cmex10]{amsmath}

\usepackage{amsthm}

\usepackage{mathrsfs}
\usepackage{txfonts}
\usepackage{stfloats}
\usepackage{bm}
\usepackage{cite}
\usepackage{cases}
\usepackage{subfig}

\usepackage{longtable}
\usepackage{multirow}

\usepackage{enumitem}
\usepackage{mathtools}
\usepackage{steinmetz}
\usepackage{tikz}
\usepackage{circuitikz}
\usepackage{verbatim}
\usepackage{tfrupee}
\usepackage[breaklinks=true]{hyperref}
\usepackage{graphicx}
\usepackage{tkz-euclide}

\usetikzlibrary{calc,math}
\usepackage{listings}
    \usepackage{color}                                            %%
    \usepackage{array}                                            %%
    \usepackage{longtable}                                        %%
    \usepackage{calc}                                             %%
    \usepackage{multirow}                                         %%
    \usepackage{hhline}                                           %%
    \usepackage{ifthen}                                           %%
    \usepackage{lscape}     
\usepackage{multicol}
\usepackage{chngcntr}

\DeclareMathOperator*{\Res}{Res}

\renewcommand\thesection{\arabic{section}}
\renewcommand\thesubsection{\thesection.\arabic{subsection}}
\renewcommand\thesubsubsection{\thesubsection.\arabic{subsubsection}}
\newtheorem{theorem}{Theorem}[section]
\newtheorem{corollary}{Corollary}[theorem]
\newtheorem{lemma}[theorem]{Lemma}
\newtheorem{definition}{Definition}[section]
\renewcommand\thesectiondis{\arabic{section}}
\renewcommand\thesubsectiondis{\thesectiondis.\arabic{subsection}}
\renewcommand\thesubsubsectiondis{\thesubsectiondis.\arabic{subsubsection}}


\hyphenation{op-tical net-works semi-conduc-tor}
\def\inputGnumericTable{}                                 %%

\lstset{
%language=C,
frame=single, 
breaklines=true,
columns=fullflexible
}
\begin{document}

\newcommand{\BEQA}{\begin{eqnarray}}
\newcommand{\EEQA}{\end{eqnarray}}
\newcommand{\define}{\stackrel{\triangle}{=}}
\bibliographystyle{IEEEtran}
\raggedbottom
\setlength{\parindent}{0pt}
\providecommand{\mbf}{\mathbf}
\providecommand{\pr}[1]{\ensuremath{\Pr\left(#1\right)}}
\providecommand{\qfunc}[1]{\ensuremath{Q\left(#1\right)}}
\providecommand{\sbrak}[1]{\ensuremath{{}\left[#1\right]}}
\providecommand{\lsbrak}[1]{\ensuremath{{}\left[#1\right.}}
\providecommand{\rsbrak}[1]{\ensuremath{{}\left.#1\right]}}
\providecommand{\brak}[1]{\ensuremath{\left(#1\right)}}
\providecommand{\lbrak}[1]{\ensuremath{\left(#1\right.}}
\providecommand{\rbrak}[1]{\ensuremath{\left.#1\right)}}
\providecommand{\cbrak}[1]{\ensuremath{\left\{#1\right\}}}
\providecommand{\lcbrak}[1]{\ensuremath{\left\{#1\right.}}
\providecommand{\rcbrak}[1]{\ensuremath{\left.#1\right\}}}
\theoremstyle{remark}
\newtheorem{rem}{Remark}
\newcommand{\sgn}{\mathop{\mathrm{sgn}}}
\providecommand{\abs}[1]{\vert#1\vert}
\providecommand{\res}[1]{\Res\displaylimits_{#1}} 
\providecommand{\norm}[1]{\lVert#1\rVert}
%\providecommand{\norm}[1]{\lVert#1\rVert}
\providecommand{\mtx}[1]{\mathbf{#1}}
\providecommand{\mean}[1]{E[ #1 ]}
\providecommand{\fourier}{\overset{\mathcal{F}}{ \rightleftharpoons}}
%\providecommand{\hilbert}{\overset{\mathcal{H}}{ \rightleftharpoons}}
\providecommand{\system}{\overset{\mathcal{H}}{ \longleftrightarrow}}
	%\newcommand{\solution}[2]{\textbf{Solution:}{#1}}
\newcommand{\solution}{\noindent \textbf{Solution: }}
\newcommand{\cosec}{\,\text{cosec}\,}
\providecommand{\dec}[2]{\ensuremath{\overset{#1}{\underset{#2}{\gtrless}}}}
\newcommand{\myvec}[1]{\ensuremath{\begin{pmatrix}#1\end{pmatrix}}}
\newcommand{\mydet}[1]{\ensuremath{\begin{vmatrix}#1\end{vmatrix}}}
\numberwithin{equation}{subsection}
\makeatletter
\@addtoreset{figure}{problem}
\makeatother
\let\StandardTheFigure\thefigure
\let\vec\mathbf
\renewcommand{\thefigure}{\theproblem}
\def\putbox#1#2#3{\makebox[0in][l]{\makebox[#1][l]{}\raisebox{\baselineskip}[0in][0in]{\raisebox{#2}[0in][0in]{#3}}}}
     \def\rightbox#1{\makebox[0in][r]{#1}}
     \def\centbox#1{\makebox[0in]{#1}}
     \def\topbox#1{\raisebox{-\baselineskip}[0in][0in]{#1}}
     \def\midbox#1{\raisebox{-0.5\baselineskip}[0in][0in]{#1}}
\vspace{3cm}
\title{Assignment 8}
\author{Adarsh Sai - AI20BTECH11001}
\maketitle
\newpage
\bigskip
\renewcommand{\thefigure}{\theenumi}
\renewcommand{\thetable}{\theenumi}
Download all python codes from 
\begin{lstlisting}
https://github.com/Adarsh541/AI1103-prob-and-ranvar/blob/main/Assignment8.1/codes/Assignment8.1.py
\end{lstlisting}
%
and latex-tikz codes from 
%
\begin{lstlisting}
https://github.com/Adarsh541/AI1103-prob-and-ranvar/blob/main/Assignment8.1/Assignment8.1.tex
\end{lstlisting}
\section{Problem}
Let $X_1,X_2,X_3,X_4,X_5$ be a random sample of size 5 from a population having standard normal distribution.If 
$\overline{X}=\frac{1}{5}\sum_{i=1}^5 X_i$ and $T=\sum_{i=1}^5\brak{X_i-\overline{X}}^2$
then $\mean{T^2\overline{X}^2}$ is equal to 
\begin{enumerate}
    \item 3
    \item 3.6
    \item 4.8
    \item 5.2
\end{enumerate}
\section{Solution}
Let $\vec{x} \sim N\brak{\vec{0},\vec{I}}$ be a standard normal random vector 
\begin{align}
    \vec{x}=\myvec{X_1\\
             X_2\\
             X_3\\
             X_4\\
             X_5}
\end{align}
Then $\overline{X}$ can be written as
\begin{align}
    \overline{X}=\frac{1}{5}\vec{u}^{\top}\vec{x}
\end{align}
where 
\begin{align}
    \vec{u}=\myvec{1\\
                      1\\
                      1\\
                      1\\
                      1}
\end{align} 
\begin{align}
   T=\vec{x}^{\top}\vec{M}\vec{x}
\end{align}
where
\begin{align}
    \vec{M}&=\myvec{\frac{4}{5}&-1/5&-1/5&-1/5&-1/5\\
                   -1/5&\frac{4}{5}&-1/5&-1/5&-1/5\\
                   -1/5&-1/5&\frac{4}{5}&-1/5&-1/5\\
                   -1/5&-1/5&-1/5&\frac{4}{5}&-1/5\\
                   -1/5&-1/5&-1/5&-1/5&\frac{4}{5}}\\
\end{align}
we also have 
\begin{align}
    \vec{M}^2&=\vec{M}
\end{align}
\begin{lemma}
\begin{align}
   \brak{\vec{u}^{\top}\vec{x}}\brak{\vec{x}^{\top}\vec{v}}=\vec{u}^{\top}\brak{\vec{x}\vec{x}^{\top}}\vec{v}
\end{align}
This lemma is verified using python simulation.
\end{lemma}
\begin{definition}[cross-covariance]
\begin{align}
    Cov\sbrak{\vec{x},\vec{y}}=\mean{\brak{\vec{x}-\mean{\vec{x}}}\brak{\vec{y}-\mean{\vec{y}}}^{\top}}
\end{align}
\end{definition}
\begin{theorem}
Let $\overline{X}$ be the sample mean of size 5 from a standard normal distribution.Then
\begin{enumerate}
    \item $\overline{X} \sim N(0,\frac{1}{5})$
    \item $\overline{X}$ and $T$ are independent.
    \item $T \sim \chi_{4}^2$
\end{enumerate} 
where $\chi_{4}^2$  is  chi-square distribution
with   $4$ degrees of freedom and $T$ is defined as
\begin{align}
    T=\sum_{i=1}^{5}(X_i-\overline{X})^2
\end{align}
\end{theorem}
\begin{proof}
\begin{align}
    \overline{X}=\frac{1}{5}\vec{u}^{\top}\vec{x}\label{3.1}
\end{align}
since $\vec{M}$ is symmetric and idempotent we have
\begin{align}
   T&=\vec{x}^{\top}\vec{M}\vec{x}\\
    &=\vec{x}^{\top}\vec{M}\vec{M}\vec{x}\\
    &=\vec{x}^{\top}\vec{M}^{\top}\vec{M}\vec{x}\\
    &=\brak{\vec{M}\vec{x}}^{\top}\brak{\vec{M}\vec{x}}\label{3.5} 
\end{align}
From $\eqref{3.1}$ it can be seen that $\overline{X}$ depends only on $\vec{u}^{\top}\vec{x}$ and from $\eqref{3.5}$ it can be seen that T depends only on $\vec{M}\vec{x}$.
So if $\vec{M}\vec{x}$ and $\vec{u}^{\top}\vec{x}$ are independent then T and $\overline{X}$  are also independent.
\begin{multline}
    Cov\sbrak{\vec{u}^{\top}\vec{x},\vec{M}\vec{x}}=\mean{\brak{\vec{u}^{\top}\vec{x}-\mean{\vec{u}^{\top}\vec{x}}}\\\brak{\vec{M}\vec{x}-\mean{\vec{M}\vec{x}}}^{\top}}
\end{multline}
From the lemma stated above we get
\begin{align}
 Cov\sbrak{\vec{u}^{\top}\vec{x},\vec{M}\vec{x}}   &=\vec{u}^{\top}\mean{\brak{\vec{x}-\mean{\vec{x}}}\brak{\vec{x}-\mean{\vec{x}}}^{\top}}\vec{M}^{\top}\\
    &=\vec{u}^{\top}Var\sbrak{\vec{x}}\vec{M}\\
    &=\vec{u}^{\top}\vec{M}\\
    &=0
\end{align}
Two jointly normal vectors are independent if and only if their cross-covariance is zero.So $\vec{M}\vec{x}$ and $\vec{u}^{\top}\vec{x}$ are independent.This implies $\overline{X}$ and T are independent.
\end{proof}
\begin{align}
    \mean{T^2\overline{X}^2}=\mean{T^2}\mean{\overline{X}^2}\label{ra}
\end{align}
\begin{align}
    \mean{\overline{X}^2}&=Var\sbrak{\overline{X}}+\brak{\mean{\overline{X}}}^2\\
    &=\frac{1}{5}
\end{align}
 since T is chi-squared distributed with 4 degrees of freedom
 \begin{align}
     \mean{T}&=4\\
     Var\sbrak{T}&=8\\
     \implies \mean{T^2}&=Var\sbrak{T}+\brak{\mean{T}}^2\\
    &=24
 \end{align}
 From $\eqref{ra}$
\begin{align}
    \mean{T^2\overline{X}^2}=4.8
\end{align}
\end{document}


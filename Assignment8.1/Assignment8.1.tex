
  
\documentclass[journal,12pt,twocolumn]{IEEEtran}

\usepackage{setspace}
\usepackage{gensymb}
\singlespacing
\usepackage[cmex10]{amsmath}

\usepackage{amsthm}

\usepackage{mathrsfs}
\usepackage{txfonts}
\usepackage{stfloats}
\usepackage{bm}
\usepackage{cite}
\usepackage{cases}
\usepackage{subfig}

\usepackage{longtable}
\usepackage{multirow}

\usepackage{enumitem}
\usepackage{mathtools}
\usepackage{steinmetz}
\usepackage{tikz}
\usepackage{circuitikz}
\usepackage{verbatim}
\usepackage{tfrupee}
\usepackage[breaklinks=true]{hyperref}
\usepackage{graphicx}
\usepackage{tkz-euclide}

\usetikzlibrary{calc,math}
\usepackage{listings}
    \usepackage{color}                                            %%
    \usepackage{array}                                            %%
    \usepackage{longtable}                                        %%
    \usepackage{calc}                                             %%
    \usepackage{multirow}                                         %%
    \usepackage{hhline}                                           %%
    \usepackage{ifthen}                                           %%
    \usepackage{lscape}     
\usepackage{multicol}
\usepackage{chngcntr}

\DeclareMathOperator*{\Res}{Res}

\renewcommand\thesection{\arabic{section}}
\renewcommand\thesubsection{\thesection.\arabic{subsection}}
\renewcommand\thesubsubsection{\thesubsection.\arabic{subsubsection}}

\renewcommand\thesectiondis{\arabic{section}}
\renewcommand\thesubsectiondis{\thesectiondis.\arabic{subsection}}
\renewcommand\thesubsubsectiondis{\thesubsectiondis.\arabic{subsubsection}}


\hyphenation{op-tical net-works semi-conduc-tor}
\def\inputGnumericTable{}                                 %%

\lstset{
%language=C,
frame=single, 
breaklines=true,
columns=fullflexible
}
\begin{document}

\newcommand{\BEQA}{\begin{eqnarray}}
\newcommand{\EEQA}{\end{eqnarray}}
\newcommand{\define}{\stackrel{\triangle}{=}}
\bibliographystyle{IEEEtran}
\raggedbottom
\setlength{\parindent}{0pt}
\providecommand{\mbf}{\mathbf}
\providecommand{\pr}[1]{\ensuremath{\Pr\left(#1\right)}}
\providecommand{\qfunc}[1]{\ensuremath{Q\left(#1\right)}}
\providecommand{\sbrak}[1]{\ensuremath{{}\left[#1\right]}}
\providecommand{\lsbrak}[1]{\ensuremath{{}\left[#1\right.}}
\providecommand{\rsbrak}[1]{\ensuremath{{}\left.#1\right]}}
\providecommand{\brak}[1]{\ensuremath{\left(#1\right)}}
\providecommand{\lbrak}[1]{\ensuremath{\left(#1\right.}}
\providecommand{\rbrak}[1]{\ensuremath{\left.#1\right)}}
\providecommand{\cbrak}[1]{\ensuremath{\left\{#1\right\}}}
\providecommand{\lcbrak}[1]{\ensuremath{\left\{#1\right.}}
\providecommand{\rcbrak}[1]{\ensuremath{\left.#1\right\}}}
\theoremstyle{remark}
\newtheorem{rem}{Remark}
\newcommand{\sgn}{\mathop{\mathrm{sgn}}}
\providecommand{\abs}[1]{\vert#1\vert}
\providecommand{\res}[1]{\Res\displaylimits_{#1}} 
\providecommand{\norm}[1]{\lVert#1\rVert}
%\providecommand{\norm}[1]{\lVert#1\rVert}
\providecommand{\mtx}[1]{\mathbf{#1}}
\providecommand{\mean}[1]{E[ #1 ]}
\providecommand{\fourier}{\overset{\mathcal{F}}{ \rightleftharpoons}}
%\providecommand{\hilbert}{\overset{\mathcal{H}}{ \rightleftharpoons}}
\providecommand{\system}{\overset{\mathcal{H}}{ \longleftrightarrow}}
	%\newcommand{\solution}[2]{\textbf{Solution:}{#1}}
\newcommand{\solution}{\noindent \textbf{Solution: }}
\newcommand{\cosec}{\,\text{cosec}\,}
\providecommand{\dec}[2]{\ensuremath{\overset{#1}{\underset{#2}{\gtrless}}}}
\newcommand{\myvec}[1]{\ensuremath{\begin{pmatrix}#1\end{pmatrix}}}
\newcommand{\mydet}[1]{\ensuremath{\begin{vmatrix}#1\end{vmatrix}}}
\numberwithin{equation}{subsection}
\makeatletter
\@addtoreset{figure}{problem}
\makeatother
\let\StandardTheFigure\thefigure
\let\vec\mathbf
\renewcommand{\thefigure}{\theproblem}
\def\putbox#1#2#3{\makebox[0in][l]{\makebox[#1][l]{}\raisebox{\baselineskip}[0in][0in]{\raisebox{#2}[0in][0in]{#3}}}}
     \def\rightbox#1{\makebox[0in][r]{#1}}
     \def\centbox#1{\makebox[0in]{#1}}
     \def\topbox#1{\raisebox{-\baselineskip}[0in][0in]{#1}}
     \def\midbox#1{\raisebox{-0.5\baselineskip}[0in][0in]{#1}}
\vspace{3cm}
\title{Assignment 8}
\author{Adarsh Sai - AI20BTECH11001}
\maketitle
\newpage
\bigskip
\renewcommand{\thefigure}{\theenumi}
\renewcommand{\thetable}{\theenumi}
Download all python codes from 
\begin{lstlisting}
https://github.com/Adarsh541/AI1103-prob-and-ranvar/blob/main/Assignment8.1/codes/Assignment8.1.py
\end{lstlisting}
%
and latex-tikz codes from 
%
\begin{lstlisting}
https://github.com/Adarsh541/AI1103-prob-and-ranvar/blob/main/Assignment8.1/Assignment8.1.tex
\end{lstlisting}
\section{Problem}
Let $X_1,X_2,X_3,X_4,X_5$ be a random sample of size 5 from a population having standard normal distribution.If 
$\overline{X}=\frac{1}{5}\sum_{i=1}^5 X_i$ and $T=\sum_{i=1}^5\brak{X_i-\overline{X}}^2$
then $\mean{T^2\overline{X}^2}$ is equal to 
\begin{enumerate}
    \item 3
    \item 3.6
    \item 4.8
    \item 5.2
\end{enumerate}
\section{Solution}
\subsection{Theorem}
Let $\overline{X_n}$ be the sample mean of size n from a normal distribution with mean $\mu$ and variance $\sigma^2$.Then \begin{enumerate}
    \item $\overline{X_n} \sim N(\mu,\frac{\sigma^2}{n})$
    \item $\overline{X}$ and $S^2$ are independent.
    \item $\frac{(n-1)S^2}{\sigma^2} \sim \chi_{n-1}^2$
\end{enumerate} 
where $\chi_{n-1}^2$  is  chi-square distribution
with   $(n-1)$ degrees of freedom and $S^2$ is defined as
\begin{align}
    S^2=\frac{1}{n-1}\sum_{i=1}^{n}(X_i-\overline{X})^2
\end{align}
\subsection{Useful concepts}
If X,Y are independent random variables.then
\begin{align}
    \mean{XY}&=\mean{X}\mean{Y}\\
    \mean{X^2}&=Var\sbrak{X}+\brak{\mean{X}}^2\label{sq}
\end{align}
If X is chi-square distributed with parameter k,then
\begin{align}
    \mean{X}&=k\label{cm}\\
    Var\sbrak{X}&=2k\label{cv}
\end{align}
\subsection{solution}
For standard normal distribution $\mu=0,\sigma^2=1$.So from the above theorem for $n=5$ we have
\begin{enumerate}
    \item T/4 and $\overline{X}$ are independent.
    \item $\overline{X} \sim N\brak{0,1/5}$
    \item $T \sim \chi_4^2$
\end{enumerate}
So from $\eqref{cm}$ and $\eqref{cv}$
\begin{align}
    \mean{T}=4\\
    Var\sbrak{T}=8
\end{align}
Since $\frac{T}{4}$ and $\overline{X}$ are independent,T and $\overline{X}$ are also independent
\begin{align}
    \mean{T^2\overline{X}^2}=\mean{T^2}\mean{\overline{X}^2}\label{ra}
\end{align}
from $\eqref{sq}$ 
\begin{align}
    \mean{\overline{X}^2}&=\frac{1}{5}\\
    \mean{T^2}&=24
\end{align}
So from $\eqref{ra}$
\begin{align}
    \mean{T^2\overline{X}^2}=4.8
\end{align}
\section{Independency}
In this section we will prove that for a normally distributed population the sample mean and sample variance are independent.The sample mean is defined as
\begin{align}
    \overline{X}=\frac{1}{n}\sum_{i=1}^{n}X_i
\end{align}
the sample variance is defined as
\begin{align}
    S^2=\frac{1}{n-1}\sum_{i=1}^{n}(X_i-\overline{X})^2
\end{align}
where $X_i$ are i.i.d and normally distributed and n is the sample size.
\subsection{properties of mean and variance}
If X and Y are independent random variables
\begin{align}
    \mean{aX+bY}&=a\mean{X}+b\mean{Y}\\
    Var\sbrak{aX+b}&=a^2Var\sbrak{X}\\
    Var\sbrak{aX+bY}&=a^2Var\sbrak{X}+b^2Var\sbrak{Y}
\end{align}
\subsection{properties of covariance}
If X,Y,Z are random variables and a,b are constants
\begin{align}
  Cov\sbrak{aX,bY}&=ab\times Cov\sbrak{X,Y}\\
Cov\sbrak{X+Z,Y}&=Cov\sbrak{X,Y}+Cov\sbrak{Z,Y}\\
Cov\sbrak{X,X}&=Var\sbrak{X}
\end{align}
\subsection{Multivariate normal distribution}
One definition is that a random vector is said to be a k-variate normally distributed if every linear combination of its k components has univariate normal distribution.If a random vector has multivariate normal distribution then any two or more of its components that are uncorrelated are independent.
\subsection{proof}
$X_i$ are i.i.d and normally distributed with mean $\mu$ and variance $\sigma^2$. Consider
\begin{multline}
    \overline{X}-X_j=\frac{1}{n}\brak{X_1+....+X_{j-1}+X_{j+1}+....+X_n}\\
    -\brak{\frac{n-1}{n}}X_j
\end{multline}
from properties of mean and variance
\begin{align}
  \overline{X}-X_j  =N\brak{0,\frac{n-1}{n}\sigma^2}
\end{align}
This implies the vector $\brak{\overline{X},X_1-\overline{X},....,X_n-\overline{X}}^T$ has multivariate normal distribution.
Furthermore we have
\begin{align}
    Cov\sbrak{X_j-\overline{X},\overline{X}}&=Cov\sbrak{X_j,\overline{X}}-Cov\sbrak{\overline{X},\overline{X}}
    \label{3}
\end{align}
\begin{enumerate}
    \item
    \begin{align}
        Cov\sbrak{X_j,\overline{X}}&=Cov\sbrak{X_j,\frac{1}{n}\sum_{i=1}^n X_i}\\
       &= \frac{1}{n}\sum_{i=1}^n Cov\sbrak{X_j,X_i}
    \end{align}
Since $X_i$ are i.i.d
    \begin{align}
       Cov\sbrak{X_j,\overline{X}}&=\frac{1}{n}Cov\sbrak{X_j,X_j}\\ 
       &=\frac{\sigma^2}{n}\label{7}
    \end{align}
\end{enumerate}
So from $\eqref{3}$ and $\eqref{7}$
\begin{align}
    Cov\sbrak{X_j-\overline{X},\overline{X}}&=\frac{\sigma^2}{n}-\frac{\sigma^2}{n}\\
    &=0\label{9}
\end{align}
From $\eqref{9}$ and property on multivariate normal distribution stated above it follows that $\overline{X}$ and $X=\brak{X_1-\overline{X},....,X_n-\overline{X}}^T$ are independent normal vectors,and so $\overline{X}$ is independent of $X^TX=(n-1)S^2$.Hence for $n=5$ we can say 
\begin{align}
    \overline{X}=\frac{1}{5}\sum_{i=1}^{5}X_i
\end{align}
and
\begin{align}
    T&=4S^2\\
    &=\sum_{i=1}^{5}\brak{X_i-\overline{X}}^2
\end{align}
are independent.
\end{document}

